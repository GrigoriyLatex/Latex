\documentclass{article}
\usepackage[utf8]{inputenc}
\usepackage[T2A]{fontenc}
\usepackage[english,russian]{babel} 
\usepackage[left=25mm, top=20mm, right=25mm, bottom=30mm, nohead, nofoot]{geometry} \usepackage{amsmath,amsfonts,amssymb} % математический пакет
\usepackage{icomma} % Умная запятая
\usepackage{mathrsfs} % Красивый математический шрифт
\usepackage{fancybox,fancyhdr} 
\usepackage[usenames]{color} % отвечающий за цвет
\usepackage{colortbl} % пакеты отвечающие за цвет
\usepackage{ wasysym }% смайлик
\usepackage{ wasysym }%грустный смайлик
\pagestyle{fancy}
\fancyhf{}
\fancyfoot[R]{\thepage} 
\fancyhead[L]{14.11.2021 Спец. группа по математике 11 класс под руководительством Ердякова Григория}
\setcounter{page}{2} % счетчик нумерации страниц
\headsep=10mm 
\usepackage{xcolor}
\usepackage{hyperref} 
\hypersetup{colorlinks=true, allcolors=[RGB]{010 090 200}} % цвет ссылок 
\usepackage{graphicx}
\graphicspath{{pictures/}}
\DeclareGraphicsExtensions{.pdf,.png,.jpg}
% Абзац
\setlength{\parindent}{2em} %абзацный отступ горизонтальная длина ( em равна ширине буквы «М» текущего шрифта)
\setlength{\parskip}{4em}% расстояние между абзацем и предыдущим текстом. (ex равна высоте буквы «x» текущего шрифта)
%
\renewcommand{\baselinestretch}{1.0}
\begin{document}
 \begin{center}
     \section*{Обратные тригонометрические функции}
\end{center}
    \huge $$ 1) \ y = arcsin \ x \equiv y \in [-\frac{\pi }{2};\frac{\pi }{2}]$$ $$\text{и}$$ $$sin \ y = x$$\\ [2mm]
   $$ 2) \ y = arccos \ x \equiv y \in [0;\pi]$$ $$\text{и} $$ $$cos \ y = x$$\\[2mm]
    $$ 3) \ y = arctg \ x \equiv y \in (-\frac{\pi }{2};\frac{\pi }{2})$$ $$\text{и}$$ $$ tg \ y = x$$\\ [2mm]
    $$ 4) \ y = arcctg \ x \equiv y \in (0;\pi)$$ $$\text{и} $$ $$ctg \ y = x$$\\[2mm]

\section*{Задачи }
\huge$$ 1) \ sin(arccos\left(\frac{3}{2}\right)) = \blacksmiley$$\\[2mm]
\huge$$2) \ sin(arcctg(-1)) = \frownie$$\\[2mm]
\huge$$3) \ Решить уравнение sin(5\ arcctg \ x) = 1$$\\[2mm]
 4) Доказать, что \huge$$ arctg(x) +  arcctg (x) = \frac{\pi}{2}$$  для любых x\\[2mm]
 5) Доказать, что \huge$$ arctg(- x) = -arctg(x)$$  для любых x \\[2mm]
 6) Доказать, что \huge$$ arctg(-x) = \pi - arcctg(x)$$  для любых x 
\end{document}